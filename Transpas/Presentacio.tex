% !TEX TS-program = lualatex
\documentclass[aspectratio=149,dvipsnames,svgnames,unicode]{beamer}
\usepackage{pgfpages}
% \setbeameroption{show only notes}
% \pgfpagesuselayout{4 on 1}[a4paper,border shrink=5mm,landscape]

% Tema FSD
\usetheme{FSD}
\setbeamercovered{invisible}

% Idioma
\usepackage{polyglossia}
\setmainlanguage{catalan}
\setotherlanguage{english}

% URL's
\usepackage{url}

% Per poder tallar gràfics en inserir-los
\usepackage{graphicx}

\usepackage{multirow}
\hypersetup{
  colorlinks=true,
}

\usepackage[listings]{tcolorbox}
\usepackage{tikz}

% use our .sty file for simple movie commands
% \usepackage{pdfpc-commands}
% \usepackage{pdfpcnotes}

% Alguns Logos més
\usepackage{metalogo}

%%%%%%%%%%%%%%%%%%%%%%%%%%%%%%%%%%%%%%%%%%%%%%%%%%%%%%%%%%%%%%
% Aparença de les diapos
%%%%%%%%%%%%%%%%%%%%%%%%%%%%%%%%%%%%%%%%%%%%%%%%%%%%%%%%%%%%%%

% Info del document
\title{Introducció a Docker}
\author{Aniol Marti (amarti@caliu.cat)}
\institute[Caliu]{Caliu - Catalan Linux Users}
\date{21 de setembre de 2019}

\begin{document}

% The title frame
% Té l'opció ``plain'' per tal que les decoracions de la resta de transparències no surtin
% L'estil està definit a la plantilla corresponent del ``inner theme''
\begin{frame}[plain]
   \titlepage
\end{frame}

\begin{frame}
\includegraphics[scale=0.6]{Imatges/birth}
\end{frame}

\section*{Index}
\begin{frame}{Índex}
	\tableofcontents
\end{frame}

\section{Concepte de contenidor}
\begin{frame}
  \frametitle{Concepte de contenidor}
  
  \begin{itemize}
    \item Una unitat de programari.
    \item Un mateix entorn.
    \item Fiable.
    \item Processos aïllats.
  \end{itemize}
  
\end{frame}

\section{El paper de Docker}
\begin{frame}
  \frametitle{El paper de Docker}
  \begin{itemize}
  	\item Desenvolupar.
  	\item Desplegar.
  	\item Executar.
  \end{itemize}
\end{frame}

\section{Imatges i contenidors}
\begin{frame}
  \frametitle{Volums, imatges i contenidors.}
  Bar
\end{frame}

\section{Dockerfile}
\input{Seccions/section4.tex}
\section{Ordres bàsiques}
\begin{frame}[containsverbatim]
  \frametitle{Ordres bàsiques}
  \tiny
  \begin{lstlisting}
  # Crear una imatge
  docker build --tag=<nom_imatge> .
  
  # Veure les imatges
  docker image ls
  
  # Aixecar un contenidor
  docker run -it -v <dir_host>:<dir_container> <nom_imatge>
  
  # Veure els contenidors en marxa
  docker ps
  
  # Aturar un contenidor
  docker stop <container_id>
  
  # Iniciar un contenidor
  docker start <container_id>
  
  # Eliminar els contenidors aturats
  docker container prune
  
  # Eliminar una imatge
  docker image rm <id_imatge>
  
  \end{lstlisting}
\end{frame}

\section{Exemple pràctic}

\section*{Final}
\begin{frame}
  \frametitle{Enllaços d'interès}
  Docker: \href{https://www.docker.com/}{https://www.docker.com/} \\
  
  DockerHub: \href{https://hub.docker.com/}{https://hub.docker.com/} \\
  
  Go: \href{https://golang.org/}{https://golang.org/}\\
  
  \hrulefill
  
  Presentació: \href{https://github.com/aniolm9/docker-dlp2019}{https://github.com/aniolm9/docker-dlp2019}
\end{frame}

\end{document}
